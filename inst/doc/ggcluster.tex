
%\VignetteIndexEntry{Introduction to ggcluster}
%\VignettePackage{ggcluster}

% Definitions
\newcommand{\code}[1]{\textt{#1}}
\newcommand{\ggcluster}{\code{ggcluster}}


\documentclass[10pt,oneside]{article}

\usepackage{Sweave}
\begin{document}
\pagestyle{empty}

\setlength{\baselineskip}{1.25em}
\setlength{\parskip}{0.5em}
\setlength{\parindent}{0.0em}

%\begin{titlepage}
\title{Introduction to the \ggcluster{} package for plotting cluster analysis results}
\author{Andrie de Vries}
%\end{titlepage}

\maketitle

\ggcluster{} is a package that makes it easy to extract results of cluster analysis into a data frame.  It provides a common framework for a number of hierarchical and other cluster analysis techniques, including:
\begin{itemize}
\item kmeans() - k-means clustering
\item Mclust() - model based clustering
\end{itemize}

\section{Introduction}

Most of the cluster analysis functions in R, e.g. \code{kmeans()} and \code{dendrogram()} provide \code{plot()} functions to display the results.  However, it is sometimes hard to extract the data from this analysis to customise these plots, especially when the \code{plot()} functions don't provide an option to return the data.

The \ggluster{} package provides a general framework to extract the plot data for a variety of cluster analysis functions.

It does this by providing generic function \clusterdata{} that will extract the appropriate cluster data as well as labels.  This data is returned as a list of data.frames.  These data frames can 
  
\begin{Schunk}
\begin{Sinput}
> library(ggplot2)
> library(ggcluster)
\end{Sinput}
\end{Schunk}



\section{Cluster analysis}

Some text here.

\section{Final thoughts}

The last word.


% Start a new page
% Not echoed, not evaluated
% ONLY here for checkVignettes so that all output doesn't
% end up on one enormous page

\end{document}




